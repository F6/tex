%!TEX TS-program = xelatex

%!TEX encoding = UTF-8 Unicode

%!TEX program=xelatex
\documentclass{article}

% import external packages
% use A4 paper and moderate margin
\usepackage[a4paper]{geometry}

%     % use this settings to get rid of all the margins.
%     % the paper size is set to a5, everything else is
%     % basically unchanged.
%     % (identicle to cut out blank spaces around the 
%     % text after printing with original a4 paper)
%     \usepackage[a5paper,
%                   hmargin=1.0cm, %horizontal margin = 1 cm on each side
%                   vmargin=2.0cm, %vertical margin = 1 cm top and bottom
% %                tmargin=1.2cm, %top margin = 1.2 cm
% %                bmargin=0.8cm, %bottom margin = 0.8 cm 
% 				  ]{geometry}

% for selecting font
\usepackage[slantfont,boldfont]{xeCJK}
\usepackage{fontspec}

% all sorts of symbols and maths and units and physics
\usepackage{physics}
\usepackage{amsmath}
\usepackage{amssymb}
\usepackage{siunitx}


% mathematical proofs
\usepackage{amsthm}
\newtheorem{theorem}{Theorem}
\newtheorem{lemma}{Lemma}
\newtheorem{algorithm}{Algorithm}


\usepackage{parskip}
% \setlength{\parindent}{0cm}

% To insert HTTP hyperlinks and also mailto: hyperlinks
\usepackage{hyperref}

% We use this package to insert code pieces to the document. This is better than verbatim env because it preserves all indentations and it can also give line numbers
\usepackage{listings}
% The default font looks strange. We will just use the same font as \texttt
\lstset{
	basicstyle=\ttfamily,
	columns=fullflexible,
}

% \( \mathds R \) to denote Real domain
\usepackage{dsfont}

% In order to insert framed paragraphs (mainly for comments and notes)
\usepackage{framed}

% we want to get some simple appendices
\usepackage[toc,page]{appendix}

% Use BibTeX to manage citations
\usepackage[super,square]{natbib}

% we also need some colored text
\usepackage{xcolor}

% for a hint on progress, use the \todo command
% the benefit of this is that when compiling, we can 
% get notified if there's still work to do
\newcommand{\todo}[1]{
	{\color{red}{\textbf{[TODO]}: #1 \textbf{[/TODO]}}}
	\PackageWarning{TODO}{Text marked as todo}
}

% in Chinese documents, a new paragraph is indicated by indenting width of 2 
% Chinese chars at the beginning. This can be done with 
%  \setlength{\parindent}{21pt}  when we use a 10.5 pt font. 
% However, the parindent indents every new paragraph, which is obviously not 
% favorable for us because when we insert an equation, we typically do not want 
% to start a new paragraph, so the next lines should not be indented.
% I think the most robust way to indent the paragraphs for now is to use a 
% simple command at the begining of every new paragraph.
% we create a new command here because if we want to change the indentation 
% length later, we can always just modify this command.
% cpindent = Chinese paragraph indent
\newcommand{\cpindent}[0]{\qquad}

% use \includegraphics to insert pictures
\usepackage{graphicx}

% after \author, add \affil to indicate the affiliation
\usepackage{authblk}

% for setting line spaces, \onehalfspacing, \doublespacing
\usepackage{setspace}


% import settings for packages and document
% Set this to 4 to number \subsubsection{}
\setcounter{secnumdepth}{4}

% Set line spacing to 1.5 for the whole document
\usepackage{setspace}
\onehalfspacing
% \doublespacing

% define glossaries and abbreviations used in document (to avoid collisions)
% This file defines glossaries and abbreviations for the document.

% ===================== BEGIN GLOSSARIES AND ABBREVIATIONS =====================

\newglossaryentry{PIMD}
{
  name=Path Integral Molecular Dynamics,
  description={is a technique used to sample to quantum canonical ensemble},
  plural=Path Integral Molecular Dynamics,
}


% ====================== END GLOSSARIES AND ABBREVIATIONS ======================
\makeglossaries


\title{Example Title}
\author{Zhi Zi}
\date{\today}
    
\begin{document}
\maketitle

This is some preface before table of content

\tableofcontents

something before first section

\section{Example section}

example text1

\subsection{Example subsection}

example text2

\subsubsection{Example subsubsection}

example text3

\paragraph{Example paragraph}

example text4

\subparagraph{Example subparagraph}

example text5

\section{Math}

This is numbered equation

\begin{equation}
	E = m c^2
\end{equation}

This is inline math: \( E = mc^2 \)

This is unnumbered equation

\begin{equation*}
	E = mc^2
\end{equation*}

This is aligned math

\begin{equation}
	\begin{aligned}
		E    & = mc^2 \\
		mc^2 & = E    \\
	\end{aligned}
\end{equation}

\section{Source code}

we can use verbatim for short source code

\begin{verbatim}
    very short source code
\end{verbatim}

Note tha verbatim ignores tab but does not ignore spaces, but a lot of formatters automatically use tab with latex, so it will mess up with our indentations.
So we never use verbatim for longer pieces of code.

Instead, we use listings package

\lstinputlisting[label={src:example.cpp}, caption={Example source code (C++)},breaklines=true, frame=single, numbers=left, language=C++]{src/example.cpp}

This way the source code is nicely formatted, and also numbered.
Spaces, indentations, page breaks and breaklines are handled correctly.

We can also refer to the source codes \ref{src:example.cpp} through the ref command.

\section{Figures}

Generally I scale all figures to the width of the column, and if i want anything smaller than that, i add blank spaces around the figure instead of changing the layout of the document.

\begin{figure}
	\begin{center}
		\includegraphics[width=0.75\columnwidth]{figs/example.png}
	\end{center}
	\caption{Example Figure}
	\label{fig:example.png}
\end{figure}

refer to Fig. \ref{fig:example.png} with ref command, too

\section{Other commonly used structures}

\subsection{Enumerate}

\begin{enumerate}
	\item 12123
	\item 23234
	\item 34345
\end{enumerate}

\subsection{Todo}

\todo{I have also defined the todo macro. The macro marks texts with red color, and throws a warning when compiled}


Before the references section, this is a example reference\cite{craig2004quantum}

\bibliography{references}

\bibliographystyle{plain}


\newpage

\begin{appendices}

\section{Example Appendix}

just some example text

\end{appendices}


\end{document}
