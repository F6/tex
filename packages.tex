% use A4 paper and moderate margin
\usepackage[a4paper]{geometry}

%     % use this settings to get rid of all the margins.
%     % the paper size is set to a5, everything else is
%     % basically unchanged.
%     % (identicle to cut out blank spaces around the 
%     % text after printing with original a4 paper)
%     \usepackage[a5paper,
%                   hmargin=1.0cm, %horizontal margin = 1 cm on each side
%                   vmargin=2.0cm, %vertical margin = 1 cm top and bottom
% %                tmargin=1.2cm, %top margin = 1.2 cm
% %                bmargin=0.8cm, %bottom margin = 0.8 cm 
% 				  ]{geometry}

% for selecting font
\usepackage[slantfont,boldfont]{xeCJK}
\usepackage{fontspec}

% all sorts of symbols and maths and units and physics
\usepackage{physics}
\usepackage{amsmath}
\usepackage{amssymb}
\usepackage{siunitx}


% mathematical proofs
\usepackage{amsthm}
\newtheorem{theorem}{Theorem}
\newtheorem{lemma}{Lemma}
\newtheorem{algorithm}{Algorithm}


\usepackage{parskip}
% \setlength{\parindent}{0cm}

% To insert HTTP hyperlinks and also mailto: hyperlinks
\usepackage{hyperref}

% We use this package to insert code pieces to the document. This is better than verbatim env because it preserves all indentations and it can also give line numbers
\usepackage{listings}
% The default font looks strange. We will just use the same font as \texttt
\lstset{
	basicstyle=\ttfamily,
	columns=fullflexible,
}

% \( \mathds R \) to denote Real domain
\usepackage{dsfont}

% In order to insert framed paragraphs (mainly for comments and notes)
\usepackage{framed}

% we want to get some simple appendices
\usepackage[toc,page]{appendix}

% Use BibTeX to manage citations
\usepackage[super,square]{natbib}

% we also need some colored text
\usepackage{xcolor}

% for a hint on progress, use the \todo command
% the benefit of this is that when compiling, we can 
% get notified if there's still work to do
\newcommand{\todo}[1]{
	{\color{red}{\textbf{[TODO]}: #1 \textbf{[/TODO]}}}
	\PackageWarning{TODO}{Text marked as todo}
}

% in Chinese documents, a new paragraph is indicated by indenting width of 2 
% Chinese chars at the beginning. This can be done with 
%  \setlength{\parindent}{21pt}  when we use a 10.5 pt font. 
% However, the parindent indents every new paragraph, which is obviously not 
% favorable for us because when we insert an equation, we typically do not want 
% to start a new paragraph, so the next lines should not be indented.
% I think the most robust way to indent the paragraphs for now is to use a 
% simple command at the begining of every new paragraph.
% we create a new command here because if we want to change the indentation 
% length later, we can always just modify this command.
% cpindent = Chinese paragraph indent
\newcommand{\cpindent}[0]{\qquad}

% use \includegraphics to insert pictures
\usepackage{graphicx}

% after \author, add \affil to indicate the affiliation
\usepackage{authblk}

% for setting line spaces, \onehalfspacing, \doublespacing
\usepackage{setspace}

% for importing csv tables into the document
\usepackage{csvsimple}
\usepackage{booktabs}
\usepackage{longtable}
